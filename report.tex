\documentclass[english,a4paper,]{report}
\usepackage{lmodern}
\usepackage{amssymb,amsmath}
\usepackage{ifxetex,ifluatex}
\usepackage{fixltx2e} % provides \textsubscript
\ifnum 0\ifxetex 1\fi\ifluatex 1\fi=0 % if pdftex
  \usepackage[T1]{fontenc}
  \usepackage[utf8]{inputenc}
\else % if luatex or xelatex
  \ifxetex
    \usepackage{mathspec}
  \else
    \usepackage{fontspec}
  \fi
  \defaultfontfeatures{Ligatures=TeX,Scale=MatchLowercase}
\fi
% use upquote if available, for straight quotes in verbatim environments
\IfFileExists{upquote.sty}{\usepackage{upquote}}{}
% use microtype if available
\IfFileExists{microtype.sty}{%
\usepackage[]{microtype}
\UseMicrotypeSet[protrusion]{basicmath} % disable protrusion for tt fonts
}{}
\PassOptionsToPackage{hyphens}{url} % url is loaded by hyperref
\usepackage[unicode=true]{hyperref}
\hypersetup{
            pdftitle={Drug prescription analysis by NHS practices},
            pdfauthor={Ricard Solé Casas},
            pdfborder={0 0 0},
            breaklinks=true}
\urlstyle{same}  % don't use monospace font for urls
\usepackage[margin=1in]{geometry}
\ifnum 0\ifxetex 1\fi\ifluatex 1\fi=0 % if pdftex
  \usepackage[shorthands=off,main=english]{babel}
\else
  \usepackage{polyglossia}
  \setmainlanguage[]{english}
\fi
\usepackage{color}
\usepackage{fancyvrb}
\newcommand{\VerbBar}{|}
\newcommand{\VERB}{\Verb[commandchars=\\\{\}]}
\DefineVerbatimEnvironment{Highlighting}{Verbatim}{commandchars=\\\{\}}
% Add ',fontsize=\small' for more characters per line
\newenvironment{Shaded}{}{}
\newcommand{\KeywordTok}[1]{\textcolor[rgb]{0.00,0.44,0.13}{\textbf{#1}}}
\newcommand{\DataTypeTok}[1]{\textcolor[rgb]{0.56,0.13,0.00}{#1}}
\newcommand{\DecValTok}[1]{\textcolor[rgb]{0.25,0.63,0.44}{#1}}
\newcommand{\BaseNTok}[1]{\textcolor[rgb]{0.25,0.63,0.44}{#1}}
\newcommand{\FloatTok}[1]{\textcolor[rgb]{0.25,0.63,0.44}{#1}}
\newcommand{\ConstantTok}[1]{\textcolor[rgb]{0.53,0.00,0.00}{#1}}
\newcommand{\CharTok}[1]{\textcolor[rgb]{0.25,0.44,0.63}{#1}}
\newcommand{\SpecialCharTok}[1]{\textcolor[rgb]{0.25,0.44,0.63}{#1}}
\newcommand{\StringTok}[1]{\textcolor[rgb]{0.25,0.44,0.63}{#1}}
\newcommand{\VerbatimStringTok}[1]{\textcolor[rgb]{0.25,0.44,0.63}{#1}}
\newcommand{\SpecialStringTok}[1]{\textcolor[rgb]{0.73,0.40,0.53}{#1}}
\newcommand{\ImportTok}[1]{#1}
\newcommand{\CommentTok}[1]{\textcolor[rgb]{0.38,0.63,0.69}{\textit{#1}}}
\newcommand{\DocumentationTok}[1]{\textcolor[rgb]{0.73,0.13,0.13}{\textit{#1}}}
\newcommand{\AnnotationTok}[1]{\textcolor[rgb]{0.38,0.63,0.69}{\textbf{\textit{#1}}}}
\newcommand{\CommentVarTok}[1]{\textcolor[rgb]{0.38,0.63,0.69}{\textbf{\textit{#1}}}}
\newcommand{\OtherTok}[1]{\textcolor[rgb]{0.00,0.44,0.13}{#1}}
\newcommand{\FunctionTok}[1]{\textcolor[rgb]{0.02,0.16,0.49}{#1}}
\newcommand{\VariableTok}[1]{\textcolor[rgb]{0.10,0.09,0.49}{#1}}
\newcommand{\ControlFlowTok}[1]{\textcolor[rgb]{0.00,0.44,0.13}{\textbf{#1}}}
\newcommand{\OperatorTok}[1]{\textcolor[rgb]{0.40,0.40,0.40}{#1}}
\newcommand{\BuiltInTok}[1]{#1}
\newcommand{\ExtensionTok}[1]{#1}
\newcommand{\PreprocessorTok}[1]{\textcolor[rgb]{0.74,0.48,0.00}{#1}}
\newcommand{\AttributeTok}[1]{\textcolor[rgb]{0.49,0.56,0.16}{#1}}
\newcommand{\RegionMarkerTok}[1]{#1}
\newcommand{\InformationTok}[1]{\textcolor[rgb]{0.38,0.63,0.69}{\textbf{\textit{#1}}}}
\newcommand{\WarningTok}[1]{\textcolor[rgb]{0.38,0.63,0.69}{\textbf{\textit{#1}}}}
\newcommand{\AlertTok}[1]{\textcolor[rgb]{1.00,0.00,0.00}{\textbf{#1}}}
\newcommand{\ErrorTok}[1]{\textcolor[rgb]{1.00,0.00,0.00}{\textbf{#1}}}
\newcommand{\NormalTok}[1]{#1}
% Make links footnotes instead of hotlinks:
\renewcommand{\href}[2]{#2\footnote{\url{#1}}}
\IfFileExists{parskip.sty}{%
\usepackage{parskip}
}{% else
\setlength{\parindent}{0pt}
\setlength{\parskip}{6pt plus 2pt minus 1pt}
}
\setlength{\emergencystretch}{3em}  % prevent overfull lines
\providecommand{\tightlist}{%
  \setlength{\itemsep}{0pt}\setlength{\parskip}{0pt}}
\setcounter{secnumdepth}{5}

% set default figure placement to htbp
\makeatletter
\def\fps@figure{htbp}
\makeatother

\usepackage{minted}
\usemintedstyle{autumn}

\usepackage{fontspec}
\setmonofont{Hasklig}
\defaultfontfeatures{Mapping=tex-text,Scale=MatchLowercase,Ligatures=TeX}

\title{Drug prescription analysis by NHS practices}
\author{Ricard Solé Casas}
\providecommand{\institute}[1]{}
\institute{Google UK \and Ada National College for Digital Skills}
\date{\today}

\begin{document}
\maketitle

\vspace*{\fill}

\section*{Summary}

This report contains a brief analysis on drug prescription by NHS
practices during the period of January and February of 2016.

\section*{Declaration}

I confirm that the submitted coursework is my own work and that all
material attributed to others (whether published or unpublished) has
been clearly identified and fully acknowledged and referred to original
sources. I agree that the College has the right to submit my work to the
plagiarism detection service. TurnitinUK for originality checks.

\section*{Acknowledgements}

I'd like to thank my partner Shannon for her continued support and
challenges that help me grow, both professionally and personally. I
would also like to thank all of you who also helped me get here.

\vspace*{\fill}

{
\setcounter{tocdepth}{2}
\tableofcontents
}
\chapter{Introduction}\label{introduction}

\section{Goal}\label{goal}

The goal of this analysis is to get a better grasp on drug prescriptions
per NHS practices and expenditure per practice over the months of
January and February of the year 2016.

\section{Tasks}\label{tasks}

We will download and load CSV files containing data for registered
chemicals, practices, prescriptions and patients over the specified time
periods.

The analysis will cover:

\begin{itemize}
\tightlist
\item
  The amount of practices in a specific area, along with its registered
  patients.
\item
  Prescription amounts for some specific medicines/chemicals.
\item
  Estimated expenditure per practice in prescriptions.
\end{itemize}

\section{Further materials and
remarks}\label{further-materials-and-remarks}

The entire source code for this report in markdown and latex formats,
and the SQL queries to load data, create views and analyze the data that
are use in the report are publicly published on
\href{https://github.com/rcsole/coursework-db}{Github} under the
\href{https://choosealicense.com/licenses/mit/}{MIT License}.

\subsection{Running}\label{running}

A utility script is provided in the Github project to \texttt{setup} and
\texttt{run-queries} on your local machine:

\begin{quote}
\textbf{Disclaimer:} \emph{Assumes bash CLI, MariaDB w/ password and a
working internet connection.}
\end{quote}

\begin{Shaded}
\begin{Highlighting}[]
\ExtensionTok{./setup} \KeywordTok{&&} \ExtensionTok{./run-queries}
\end{Highlighting}
\end{Shaded}

\chapter{Starting up}\label{starting-up}

For consistency with the lectures the RDMS used is
\href{https://mariadb.org/}{MariaDB}, a fork of
\href{https://mysql.com}{MySQL}. The OS is Apple's
\href{https://www.wikiwand.com/en/MacOS}{OSX}, and the package manager
to install MariaDB we used for this exercise is
\href{https://brew.sh}{homebrew}. I chose MariaDB over MySQL because
I'll be working at Google and that seems to be their fork.

\section{Installing MariaDB}\label{installing-mariadb}

Using the \texttt{brew} command, in our terminal:

\begin{Shaded}
\begin{Highlighting}[]
\ExtensionTok{brew}\NormalTok{ --version}
\ExtensionTok{Homebrew}\NormalTok{ 1.2.2}
\ExtensionTok{Homebrew/homebrew-core}\NormalTok{ (git revision 73a8655}\KeywordTok{;} \FunctionTok{last}\NormalTok{ commit 2017-06-12)}

\ExtensionTok{brew}\NormalTok{ install mariadb}
\end{Highlighting}
\end{Shaded}

We might be prompted to \emph{secure} the \texttt{MySQL/MariaDB} (we
will use both names interchangeably from now on), this is accomplished
by following the steps asked in this command:

\begin{Shaded}
\begin{Highlighting}[]
\ExtensionTok{mysql_secure_installation}
\end{Highlighting}
\end{Shaded}

\section{Creating the database}\label{creating-the-database}

With some really simple steps we'll be able to create a database.
Fortunately, I've create a bash script that will automate the setup on a
UNIX system. Here, however, we'll describe it step by step.

\begin{Shaded}
\begin{Highlighting}[]
\ExtensionTok{mysql}\NormalTok{ -uroot -p}
\end{Highlighting}
\end{Shaded}

\begin{verbatim}
Welcome to the MariaDB monitor.  Commands end with ; or \g.
Your MariaDB connection id is 78
Server version: 10.2.6-MariaDB Homebrew

Copyright (c) 2000, 2017, Oracle, MariaDB Corporation Ab and others.

Type 'help;' or '\h' for help. Type '\c' to clear the current input statement.

MariaDB [(none)]> CREATE DATABASE prescriptionsdb;
Query OK, 1 row affected (0.01 sec)
\end{verbatim}

\section{Modeling the data}\label{modeling-the-data}

Upon inspection of the data, there are \textbf{four} CSV files. Three of
them found \href{https://goo.gl/zC3afI}{here} and the last one
\href{https://goo.gl/n8XbX7}{here}. This maps to also four tables in our
database. \texttt{practices} ---containing the address, names and
postcodes of GPs---, \texttt{prescriptions} ---actual prescribed
medicine, cost, GP, and quantity---, \texttt{gppatients} ---the number
oh patients, per age, per GP---, and \texttt{chemicals} ---extended
information on medicaments that are registered by the NHS.

Everything builds on the \texttt{practices} table: both
\texttt{prescriptions} and \texttt{gppatients} have a
\texttt{practiceid} field which will be a \texttt{FOREIGN\ KEY}
reference to the \texttt{practices} table.

\subsection{Create practices table}\label{create-practices-table}

\inputminted[firstline=10,lastline=19]{sql}{src/sql/00-setup.sql}

\subsection{Create prescriptions
table}\label{create-prescriptions-table}

\inputminted[firstline=21,lastline=34]{sql}{src/sql/00-setup.sql}

\subsubsection{Add FOREIGN KEY
reference}\label{add-foreign-key-reference}

\inputminted[firstline=36,lastline=36]{sql}{src/sql/00-setup.sql}

\subsection{Create gppatients table}\label{create-gppatients-table}

\inputminted[firstline=43,lastline=46]{sql}{src/sql/00-setup.sql}

\subsubsection{Add FOREIGN KEY
reference}\label{add-foreign-key-reference-1}

\inputminted[firstline=48,lastline=48]{sql}{src/sql/00-setup.sql}

\subsection{Create chemicals table}\label{create-chemicals-table}

\inputminted[firstline=38,lastline=41]{sql}{src/sql/00-setup.sql}

\section{Wrapping up}\label{wrapping-up}

This does it for scaffolding our database and tables. In the next
chapter we'll load the CSV files, and create some useful views for
future queries.

\chapter{Setting up our data}\label{setting-up-our-data}

As described in the previous chapter we'll detail step by step how to
load the data from the CSV to our \texttt{MariaDB} instance, and into
\texttt{prescriptionsdb}.

\section{Fetching the data}\label{fetching-the-data}

The data we'll be working with is from 2016, January and February. It is
provided by the government. The easiest way to fetch it is to
\texttt{cURL} it from the terminal:

\inputminted[lastline=114]{bash}{setup}

\section{How to load a CSV file}\label{how-to-load-a-csv-file}

MariaDB provides an instruction,
\texttt{LOAD\ DATA\ {[}LOCAL{]}\ INFILE}, which lets us dump a file into
a table. You can find more information on how that command works on the
MariaDB documentation website for
\href{https://mariadb.com/kb/en/mariadb/load-data-infile/}{\texttt{LOAD\ DATA}}.

\subsection{Usage}\label{usage}

\inputminted[firstline=55,lastline=97]{sql}{src/sql/00-setup.sql}

\section{Creating views}\label{creating-views}

Before moving on to actually querying our data, I found it useful to
look at the specified tasks and possibly create views that would aid
said queries. From Wikipedia:

\begin{quote}
\emph{``In database theory, a view is the result set of a stored query
on the data, which the database users can query just as they would in a
persistent database collection object. This pre-established query
command is kept in the database dictionary.''}
\end{quote}

You can find more information on what views are on the MariaDB
documentation website
\href{https://mariadb.com/kb/en/mariadb/views/}{about views}, or on
\href{https://www.wikiwand.com/en/View_(SQL)}{Wikipedia}.

\subsection{Beta-blockers}\label{beta-blockers}

One of the first things asked is to identify
\href{http://bit.ly/2reDxBL}{beta-blocker} prescriptions. The NHS
Provides with some information in what some common beta-blockers are. I
found it particularly useful to create a view of what prescriptions
actually match the chemicals commonly identified as beta-blockers.

\inputminted[firstline=4,lastline=23]{sql}{src/sql/01-views.sql}

\subsection{Prescriptions-per-x}\label{prescriptions-per-x}

Other tasks ask for common patterns such as prescriptions per gp, and
prescriptions per chemical.

\inputminted[firstline=26,lastline=38]{sql}{src/sql/01-views.sql}

\subsection{Other views}\label{other-views}

There are some other views that proved to be necessary/useful as queries
were being constructed. These will be discussed further with their
corresponding queries.

\inputminted[firstline=41]{sql}{src/sql/01-views.sql}

\section{Wrapping up}\label{wrapping-up-1}

That's all there is to model our data. To recap we have:

\begin{itemize}
\tightlist
\item
  Downloaded the CSV files into a data folder using \texttt{cURL}.
\item
  Loaded the CSV files using \texttt{LOAD\ DATA}
\item
  Created some \texttt{VIEW}s that we'll be using later to abstract
  complexity from our queries.
\end{itemize}

\chapter{Data Analysis}\label{data-analysis}

\section{Practices in a particular
area}\label{practices-in-a-particular-area}

We are particularly interested in \emph{how} many GPs ---and how many
registered patients in said GPs--- there are in a specific area covered
by the NHS ---in N17.

Getting the amount of GPs is an easy task to accomplish with the
provided data. From the \texttt{practices} table we want to retrieve the
rows whose \texttt{postcode} column start with \texttt{N17}. However, we
also want the patients from that area. That's in the \texttt{gppatients}
table which relates to \texttt{practices} through the
\texttt{practiceid} column. In this case we will need a
\texttt{LEFT\ JOIN} when combining both tables. The reason we need it to
be \texttt{LEFT} is because we might find some \texttt{practices} which
do not have any registered patients but \textbf{are} in the \emph{N17}
postcode.

\inputminted[firstline=5,lastline=15]{sql}{src/sql/02-queries.sql}

From the result of the query we can conclude there are \textbf{7
practices in N17} and a total of \textbf{49,358} patients are registered
in that postcode.

\section{Beta-blockers}\label{beta-blockers-1}

Taking advantage of the views created in section 3.3.1 the query to find
the GP which prescribed the most beta-blockers turns out to be a rather
simple \texttt{JOIN} from that \texttt{VIEW} to the \texttt{gppatients}
table:

\inputminted[firstline=27,lastline=39]{sql}{src/sql/02-queries.sql}

The practice that has prescribed the most beta-blockers per patient
registered (in that specific practice) is \textbf{Burrswood Nursing
Home}, ID \textbf{G82651}, prescribing \textbf{161} beta-blockers per
patient.

\section{Prescriptions per
medication}\label{prescriptions-per-medication}

The relevant \texttt{VIEW} here is \texttt{ppc}. It contains how many
prescriptions there are per chemical. Doing a simple \texttt{JOIN} with
the \texttt{chemicals} table, sorting it by the dispensed amount and
\texttt{LIMIT}ing it to \texttt{1} result:

\begin{quote}
\textbf{Disclaimer}: \emph{The assumption is we are looking for the
chemical that's been dispensed most often, not necessarily the one
that's been dispensed the most as far as actual quantity goes. With the
data we could gather that's a rather difficult task (more quantity does
not equal more substance).}
\end{quote}

\inputminted[firstline=42,lastline=53]{sql}{src/sql/02-queries.sql}

The most prescribed medication in January \& February of 2016 was
\textbf{Colecalciferol}, which was prescribed \textbf{280495} times.

\section{Expenditure per practice}\label{expenditure-per-practice}

Like in every other query we also have a relevant view here,
\texttt{spentpergp}. All we need to do is join \texttt{spentpergp} with
\texttt{practices} to get the name, and we are done! \texttt{spentpergp}
is the result of \texttt{SUM}ing all \texttt{prescriptions}
\texttt{GROUP}ed \texttt{BY} their \texttt{practiceid}:

\inputminted[firstline=56,lastline=80]{sql}{src/sql/02-queries.sql}

The \textbf{biggest spender} was \textbf{Midlands Medical Partnership}
at \textbf{1,638,640.13£} per patient. The \textbf{cheapest practice}
was \textbf{Cri Bury Recovery Services} at \textbf{0.17£} per patient.

\section{SSRI prescriptions change}\label{ssri-prescriptions-change}

There are similar \texttt{VIEW}s for
\href{http://www.nhs.uk/conditions/SSRIs-(selective-serotonin-reuptake-inhibitors)/Pages/Introduction.aspx}{SSRI}
as there were for beta-blocker. To find the difference in amount of
times SSRIs have been prescribed between January and February we simply
\texttt{COUNT(*)} and \texttt{GROUP\ BY} the \texttt{period} when they
were prescribed.

To know what qualifies as SSRI we've referred to the NHS website. The
relevant excerpt:

\begin{itemize}
\tightlist
\item
  Types of SSRIs

  \begin{itemize}
  \tightlist
  \item
    citalopram
  \item
    dapoxetine
  \item
    escitalopram
  \item
    fluoxetine
  \item
    fluvoxamine
  \item
    paroxetine
  \item
    sertraline
  \end{itemize}
\end{itemize}

\inputminted[firstline=92,lastline=102]{sql}{src/sql/02-queries.sql}

The difference between February and January is minimal, at only
\textbf{486} fewer SSRIs prescribed.

\section{Metformin per practice}\label{metformin-per-practice}

You guessed it right, we are using \texttt{VIEW}s! In this case
\texttt{metforminprescriptions} which builds on \texttt{metformin}. The
latter gets all the \texttt{bnfcodesub}s belonging to \texttt{chemicals}
containing \texttt{metformin}. We use the former to \texttt{JOIN} it
with \texttt{practices} on the \texttt{practiceid\ =\ id} and
\texttt{GROUP\ BY} the \texttt{id}, while also \texttt{ORDER}ing
\texttt{BY} the amount of \texttt{prescriptions} and finally
\texttt{LIMIT} it to \texttt{10} elements:

\inputminted[firstline=107]{sql}{src/sql/02-queries.sql}

The above comment shows a nicely formatted table of the \textbf{top 10}
\texttt{practices} per metformin prescribed.

\section{Wrapping up}\label{wrapping-up-2}

This covers all the analysis we are doing on drug prescription by NHS
practices.

\chapter{Conclusion}\label{conclusion}

There a few takeaways from the analysis:

\begin{itemize}
\item
  In order to better understand drug prescription quantity how
  prescriptions are stored needs to be rethought. It's hard to conclude
  which medicine has been prescribed more overall, or which practice has
  prescribed the highest \textbf{actual} amount of {[}insert medicine
  here{]}. The chemicals do not carry any information regarding dosage
  in a way that is easily parsable.
\item
  There aren't as many patients in \emph{N17} as I expected.
\item
  Whatever \emph{beta-blockers} actually are, \emph{Burrswood Nursing
  Home} must have a lot of people needing them. At a whooping 161 bb/p
  they are leading the charge on that type of drug.
\item
  We can see massive gaps on money spent per patient, it ranges from
  \emph{0.17£} to over \emph{1M£}.
\item
  As far as SSRIs go, the fluctuation between months seems to be minimal
  (less than 1\%).
\end{itemize}

\chapter{References}\label{references}

\begin{itemize}
\item
  Health and Social Care Information Centre, Prescribing (2015):
  \emph{General Practice Prescribing Data Frequently Asked Questions}.
  Available at: http://bit.ly/2tifJNU
\item
  NHS (2016): \emph{Beta-blockers}. Available at: http://bit.ly/2reDxBL
\item
  Wikipedia (2016): \emph{View (SQL)}. Available at:
  https://en.wikipedia.org/wiki/View\_(SQL)
\item
  NHS (2015): \emph{Selective serotonin reuptake inhibitors (SSRIs)}.
  Available at: http://bit.ly/1Uym01X
\end{itemize}

\end{document}
